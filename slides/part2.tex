\documentclass[12pt]{beamer}
\usetheme{Singapore}
\usepackage{mathpazo}
\renewcommand{\rmdefault}{ppl}
\setbeamerfont{title}{family=\rmfamily}
\setbeamerfont{frametitle}{family=\rmfamily}
\usepackage[utf8]{inputenc}
\usefonttheme{professionalfonts} 
\usepackage{amsmath}
\usepackage{amsfonts}
\usepackage{amssymb}
\usepackage{xcolor}
\usepackage{listings}
\usepackage{tikz}
\usetikzlibrary{positioning,fit}

\author{Lai Hao Ran \\ \footnotesize\texttt{hrlai.ecology@gmail.com}}
\title{Application on\\joint species distribution model}
\date{} 
%\setbeamercovered{transparent} 
\setbeamertemplate{navigation symbols}{} 

\titlegraphic{%
	\centering
	\includegraphics[height=0.15\textheight]{fig/Marsden-logo-RGB-150.jpg}~%
	\hspace{1cm}
	\includegraphics[height=0.15\textheight]{fig/University_of_Canterbury_logo.png}~%
	\hspace{1cm}
	\includegraphics[height=0.15\textheight]{fig/searpp.png}
} 

\lstset{
  basicstyle=\ttfamily\scriptsize,
  breaklines=true,
  showstringspaces=false,
  frame=single,
  backgroundcolor=\color{gray!10},
  literate=
    {~}{{\textasciitilde}}1
    {_}{{\_}}1
    {^}{{\textasciicircum}}1
    {\\}{{\textbackslash}}1
    {|}{{\textbar}}1
    {<}{{\textless}}1
    {>}{{\textgreater}}1
    {=}{{=}}1
}

\newcommand{\exercise}{%
  \begin{tikzpicture}[remember picture, overlay]
    \node[anchor=north east] at (current page.north east) {
        \includegraphics[width=2.5cm]{fig/exercise.png}
    };
  \end{tikzpicture}%
}
\begin{document}

\begin{frame}
\titlepage
\end{frame}

\begin{frame}{Scaling up from GLMs to GLMMs}
\begin{center}
Joint species distribution models are just glorified\\generalised linear mixed-effect models (GLMMs)
\end{center}
\end{frame}

\begin{frame}{Single-species GLM}
\only<1-4>{
The response of species A to environment $X$ across sites.
}
\only<2>{
\begin{center}
\alert{Whiteboard: graphical way}
\end{center}}
\only<3>{
\begin{align*}
Y &\sim \text{Binomial}(N,~p) \\
\text{logit}(p) &= \alpha + \beta X
\end{align*}
}
\only<4>{
\begin{align*}
Y &\sim \text{Bernoulli}(p) \\
\text{logit}(p) &= \alpha + \beta X
\end{align*}
}
\only<5->{
The response of species A to environment $X$ across site $i$.
\begin{align*}
Y_i &\sim \text{Bernoulli}(p_i) \\
\text{logit}(p_i) &= \alpha + \beta X_i
\end{align*}
}
\uncover<6>{
\begin{center}
\alert{Whiteboard: expand to multi-species dataset}
\end{center}
}
\end{frame}

\begin{frame}{Multi-species GLMM}
The response of species $j$ to environment $X$ across site $i$.
\uncover<1->{
\begin{align*}
Y_{ij} &\sim \text{Bernoulli}(p_{ij}) \\
\text{logit}(p_{ij}) &= \alpha_i + \alpha_j + \beta_j X_i
\end{align*}
}
\vfill
\uncover<2>{
\texttt{y $\sim$ (1 | site) + (1 + X | species) + X,\\
\phantom{------}family = binomial("logit")}
}
\end{frame}

\begin{frame}{Detour 1: incorporating traits}
\textcolor{blue}{How do traits moderate} the response of species $j$ to environment $X$ across site $i$.
\pause
\begin{align*}
Y_{ij} &\sim \text{Bernoulli}(p_{ij}) \\
\text{logit}(p_{ij}) &= \alpha_i + \alpha_j~\textcolor{blue}{+ \gamma T_j} + \beta_j X_i~\textcolor{blue}{+ \kappa X_i T_j} \\
&= \alpha_i + (\alpha_j~\textcolor{blue}{+ \gamma T_j}) + (\beta_j ~\textcolor{blue}{+ \kappa T_j}) X_i
\end{align*}
\pause
\texttt{y $\sim$ (1 | site) + (1 + X | species) + X \textcolor{blue}{+ T + X:T},\\
\phantom{------}family = binomial("logit")}
\end{frame}

\begin{frame}{Detour 2: species--species correlations}
The response of species $j$ to environment $X$ across site $i$, \textcolor{blue}{while accounting for non-independence among species}.
\pause
\begin{align*}
Y_{ij} &\sim \text{Bernoulli}(p_{ij}) \\
\text{logit}(p_{ij}) &= \alpha_i + \alpha_j + \beta_j X_i~\textcolor{blue}{+ \theta_{1j} Z_{1i} + \theta_{2j} Z_{2i}} 
\end{align*}
\pause
\begin{center}
\alert{Whiteboard: link to multivariate ordination}
\end{center}
\end{frame}

\begin{frame}{Fitting JSDM (finaaaaally)}
\exercise
\begin{itemize}
\item Recommended packages (with traits and latent variables)
  \begin{itemize}
  \item \texttt{boral} (Bayesian)
  \item \texttt{gllvm} (Frequentist)
  \end{itemize}
\pause
\item But we will fit a simpler JSDM with \texttt{brms}
  \begin{itemize}
  \item Transferable skill beyond JSDMs
  \item Fast
  \item Easy-to-learn syntax
  \end{itemize}
\end{itemize}
\pause
\begin{center}
\alert{Pair up with your laptops\\
\texttt{https://github.com/hrlai/bayes-workshop}}
\end{center}
\end{frame}

\begin{frame}
Sometimes classical statistics gives up. Bayes never gives up ... so we're under more responsibility to check our models.
Andrew Gelman
\end{frame}

\begin{frame}{When should I stop?}
\begin{itemize}
\item Bayesian inference is flexible, therefore it is a rabbit hole
\item When you spend too much time building ever more complex Bayesian models, it is probably a good time simplify your questions instead
\item At which point you may find that you don't need Bayes anyways
\end{itemize}
\end{frame}

\begin{frame}
I still use frequentist approach, and everytime I return to it, I realise how much more I appreciate frequentist because of what I have learnt from Bayesian inference.
\end{frame}


\begin{frame}{Feedback}

\end{frame}

{Further reading}

\end{frame}

\end{document}